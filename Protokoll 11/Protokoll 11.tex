
\documentclass[11pt,a4paper]{article}

\usepackage[utf8]{inputenc} 
\usepackage[T1]{fontenc} 
\usepackage{lmodern}
\usepackage{tcolorbox}


\usepackage[german]{babel}


\setlength{\parindent}{0pt}
\setlength{\parskip}{1ex plus 0.5ex minus 0.5ex}

\usepackage{amsmath} 


\usepackage{graphicx} 

\usepackage[section]{placeins}
\usepackage{booktabs}


\usepackage{hyperref}
\hypersetup{
	colorlinks,
	citecolor=red,
	filecolor=black,
	linkcolor=black,
	urlcolor=black}
\graphicspath{}

\begin{document}
	
	{
		\centering 
		\large 
		Physiklabor für Anfänger*innen \\
		Ferienpraktikum im Sommersemester 2018 \\[4mm]
		\textbf{\LARGE 
			Versuch 75: Lichtmikroskop
		} \\[3mm]
		(durchgeführt am 26.09.2018 bei Daniel Bartle) \\
		Ye Joon Kim, Marouan Zouari\\
		\today \\[10mm]
	}
	\tableofcontents
\section{Einleitung}
Mit einer Lupe oder einem Mikroskop können kleine Objekte oder feine Einzelheiten stark vergrößert angesehen werden. Wie groß ein Objekt aussieht, hängt von dem Sehwinkel, $\epsilon =\arctan(B/b)$ ab. Eine Lupe lenkt das Licht so ab, sodass der Sehwinkel vergrößert wird. Dadurch wird ein vergrößerndes Bild in dem Auge geschafft. Ein Mikroskop funktioniert ähnlich, aber es kann viel stärkere Vergrößerungen schaffen. Die zwei wichtigsten Teile eines Mikroskops sind das Objektiv und das Okular



\section{Aufbau}

\section{Durchführung}

\section{Auswertung und Fehleranalyse}

\section{Diskussion der Ergebnisse}

\section{Literatur}

\section{Anhang}


	
	
	
	
	
\end{document}